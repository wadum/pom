\documentclass[12pt]{article}
\usepackage[utf8]{luainputenc}
\usepackage[danish,british]{babel}
\usepackage[pdftex]{graphicx}
\usepackage{listings}
\usepackage{hyperref}
\usepackage{xcolor}

\lstset{
	basicstyle=\tiny
}

\title{Individuel opgave 1}
\author{Christoffer Wadum Larsen}

\begin{document}
\maketitle

\section{Genaflevering individuel opgave 1}

Til genafleveringen er løsningen gjort interaktiv, eftersom der var en kommentar omkring input og raw\_input gik jeg ud fra at det var et krav.

Programmet har til formål at løse et andengradspolynomium på formen $ax^2+bx+c=0$.

Hvis programmet køres fra en prompt vil main() bliver afviklet. Main har til formål at modtage a, b og c som input fra brugeren.

Main() kalder herefter findRodder(a,b,c), der returnere rødderne x1 og x2. Hvis $D < 0$, findes der ingen løsninger og m.sqrt kaster en ValueError fejl. Hvis $a == 0$ er der ikke længere tale om et andengradspolynomium, funktionen kaster derfor en ZeroDivision fejl. Hvis $D == 0$ eller $D > 0$ printer main() rødderne.

Begge disse fejl bliver fanget i main()


\section{Tests}

\begin{lstlisting}
[chris@localhost pom]$ python christoffer.larsen.36.py 
Angiv a: 1
Angiv b: 1000.001
Angiv c: 1
Rødderne for andengradspolynomiumet 1.000000x²+1000.001000b+1.000000=0 er 
	 x1 = -1000.000000, x2 = -0.001000
[chris@localhost pom]$ python christoffer.larsen.36.py 
Angiv a: 5
Angiv b: -10
Angiv c: 1
Rødderne for andengradspolynomiumet 5.000000x²+-10.000000b+1.000000=0 er 
	 x1 = 1.894427, x2 = 0.105573
\end{lstlisting}

\end{document}